\documentclass[12pt]{article}

\usepackage[spanish]{babel}
\usepackage[utf8]{inputenc}
\usepackage{graphicx}
\usepackage{geometry}
\usepackage{xcolor}
\usepackage{fancyhdr}
\usepackage{lastpage}
\usepackage{pdfpages}
\usepackage{listings}

\geometry{top=25mm,left=15mm,right=15mm,a4paper}

\pagestyle{fancy}
\fancyhf{}
\lhead{Lenguajes de Programación}
\cfoot{Página \thepage\ de \pageref{LastPage}}

\graphicspath{./}

\begin{document}
\includepdf{Portada.pdf}
{\color{red} \section*{Practica 1: Implementación y análisis de automatas celulares.}}

{\color{blue} \subsection*{Parte 1. Preguntas}}
\vspace{1em}

\begin{enumerate}
    \item \textbf{¿Qué es la modelación basada en agentes?}\\
    La modelación basada en agentes es un análisis de sistemas complejos, modelar distintos fenomenos
    colectivos (muchos agentes intercatuando en un ambiente), se forman de colecciones de agentes que interactuan en el entorno o el ambiente a traves de un determinado tiempo.

    \item \textbf{¿Qué es el enfoque bottom-up?}\\
    Este tipo de efoque se basa principalmente en la construcción del modelo que es de abajo hacia arriba,
    \item \textbf{¿Cuándo se usa el concepto de autoorganización en sistemas?}\\
    Se utiliza cuando hablamos de características de los sistemas complejos y se basa principalmente en producir patrones
    de manera espontánea conforme al tiempo sin un plan determinado, los componementes se van autoorganizando aplicando un conjunto de reglas para formar estructuras
    de forma coherente.
    \item \textbf{¿Qué es una propiedad emergente?}\\
    Es aquella propiedad que se encuentra en el sistema pero no en sus componentes, sale principalmente de las iteraciones y para esto es necesario que saquemos
    información del sistema (formación de filas en el sistema de transporte coectivo METRO.)
\end{enumerate}

\end{document}