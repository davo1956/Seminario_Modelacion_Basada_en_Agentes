\documentclass[12pt]{article}

\usepackage[spanish]{babel}
\usepackage[utf8]{inputenc}
\usepackage{graphicx}
\usepackage{geometry}
\usepackage{xcolor}
\usepackage{fancyhdr}
\usepackage{lastpage}
\usepackage{pdfpages}
\usepackage{listings}

\geometry{top=25mm,left=15mm,right=15mm,a4paper}

\pagestyle{fancy}
\fancyhf{}
\lhead{Lenguajes de Programación}
\cfoot{Página \thepage\ de \pageref{LastPage}}

\graphicspath{./}

\begin{document}
\includepdf{Portada.pdf}
{\color{red} \section*{Practica 2: Análisis de modelos basados en agentes.}}

{\color{blue} \subsection*{Parte 1. Modelo de segregación de Schelling.}}
\vspace{1em}

El modelo propuesto originalmente por Thomas Schelling consiste en dos grupos de agentes, por ejemplo rojos y verdes, que localmente tratan de satisfacer la necesidad de estar con los
de su mismo grupo. De manera general este comportamiento es establecido por un parámetro conocido como porcentaje de \textbf{similitud-requerida} o nivel de tolerancia.\\

Los agentes toman una desición apartir de la información que tienen en su vecindad. Si el agente satisface las condiciones del entorno entonces se queda en su posición actual, de lo contrario
se mueve a otra posición vacia. Esta dinamica local genera como resultado la formación de cúmmulos de agentes del mismo tipo, hay segregación.\\

\textbf{Definición del Sistema:} El sistema se compone de una reticula de $n$ x $n$ donde $n$ se establece entre $50$ y $100$.
Cada celda con posición $(i,j)$ alberga a un agente rojo o verde. El sistema tiene un parámetro de \textbf{densidad poblacional} usualmente se establece en $90\%$ (es decir, $10\%$ de las celdas quedan vacias).
La mitad de la población es color rojo y la otra mitad, color verde. Cada agente toma una celda aleatoriamente.\\

\textbf{Dinámica:} Cada agente en la posición $(i,j)$ se "muda" a un lugar vacio si su vecindad de Moore no cumple con el porcentaje de similitud requerida.\\

{\color{blue} \subsubsection*{Ejercicios:}}

\begin{enumerate}
    \item \textbf{Implmente el modelo} de segregación de Schelling original, pueden usar de base el codigo visto en clase.
    \item Establezca el tamaño de reticula como $n=50$, con densidad poblacional del $90\%$.
    \begin{enumerate}
        \item ¿Qué valor del parametro de similitud es el limite maximo para formar dinámicas de segregación? A este valor se le llamará \textbf{Smax}.\\
        \textbf{\color{red} RESPUESTA:}\\
    \end{enumerate}
    \item Una propuesta de medida para detectar convergencia es cuando los agentes ya no cambian de posición. En tiempo el sistema es: $t = (t+1)$.
    Cuando el parametro de similitud es igual a Smax.\\
    \begin{enumerate}
        \item ¿Cuál es el tiempo en el que el sistema converge?, Realice una grafica parametro-similitud vs tiempo-de-convergencia.\\
        \textbf{\color{red} RESPUESTA:}\\
        \item ¿Cómo crece el tiempo de convergencia en función del parametro de similitud?, ¿lineal, algoritmico, exponencial?. Cuando no converga el sistema (tiempo muy grande), dejar de graficar\\
        \textbf{\color{red} RESPUESTA:}\\ 
    \end{enumerate}
    \item \textbf{Extención del modelo.} Establezca el parámetro de similitud para cada uno de los agentes, como un atributo del agente. Inicialice la similitud requerida del agente i-esimo a partir de
    una distribución normal con media $50$ y desviación estandar $10$. Describa sus resultados y adjunte capturas de pantalla para dar soporte a la explicación.\\
    \begin{enumerate}
        \item ¿Cómo cambian los patrones de segregación?, Explique.
        \textbf{\color{red} RESPUESTA:}\\
        \item ¿Qué sucede cuando la media = Smax y la desviación estandar es pequeña o grande? Explique.\\
        \textbf{\color{red} RESPUESTA:}\\
    \end{enumerate}
    \item \textbf{Optativo:} Modifique su programa previo para considerar tres tipos de agentes (rojos, verdes y azules). Inicialice cada grupo como $\frac{1}{3}$ de la población y establezca de \textbf{manera global}
    el parametro de similitud requerida. Adjunte capturas de pantalla y explique la dinamica\\
    \begin{enumerate}
        \item ¿Se forman patrones de segregación?
        \item ¿Cuál es el valor del umbral Smax? 
    \end{enumerate}
    \item Bajo su criterio que otros elementos de modelación se podrian definir en el modelo de Schelling para hacerlo más realista. Explique\\
    \textbf{\color{red} RESPUESTA:}\\
    \item ¿Qué otros análisis podrian implementar para explicar las dinámicas? Explique.
\end{enumerate}

{\color{blue} \subsection*{Parte 2. Termitas Apiladoras.}}
\vspace{1em}

Este modelo fue propuesto por Mitchel Resnick como una estrategia descentralizada para apilar astillas de madera (objetos) a través de simples reglas ejecutadas por termitas (agentes).\\

\textbf{Definición del Sistema:} El sistema se compone de una reticula de $n$ x $n$ donde $n=100$. Cada celda con su posición $(i,j)$ alberga una astilla de madera (amarilla). El sistema tiene un parametro de número de termitas y densidad de astillas.\\

\textbf{Dinámica:} Las termitas tienen dos reglas basicas:\\
\begin{enumerate}
    \item Si la termita no esta cargando nada y se encuentra una astilla de madera, la recoge.
    \item Si está cargando una astilla de madera y encuentra otra, suelta la astilla y continua el camino.
\end{enumerate}
El movimiento  de la termita es un caminador aleatorio con una apertura de visión de $-50$ a $50$ grados.\\ 
{\color{blue} \subsubsection*{Ejercicios:}}

\begin{enumerate}
    \item \textbf{Implemente el modelo:} de termitas apiladoras, pueden usar el código de la biblioteca de modelos de NetLogo, si retoman el código visto en clase o lo programan en otro lenguaje de
    programación tienen un punto extra en el ejercicio.\\
    \item Implemente una grafica donde se observe el comportamiento del sistema en función del tiempo, por ejemplo, el número de cúmulos en función el tiempo, el promedio del tamaño del
    cúmulo en función del tiempo, o el número de termitas que están cargando astillas. Si proponen otra forma para obtener información, grafique y argumente porque es adecuada.\\
    \item Extender el modelo considerando dos tipos de astillas de madera (por ejemplo, amarillas y cafés). La termina deja y recoge la astilla a partir del color del cúmulo.\\
    \begin{enumerate}
        \item ¿Cuántas pilas de astillas quedan al final?. Muestre la evolución del sistema con capturas de pantalla.\\
        \textbf{\color{red} RESPUESTA:}\\
    \end{enumerate}
    \item En la regla original, la termita suelta la astilla si encuentra otra astilla del mismo color y sigue su camino. Implemente el siguiente comportamiento: una vez que suelta la astilla, la
    termita “salta” a otra posición de manera aleatoria.\\
    \begin{enumerate}
        \item Capture la pantalla de los estados finales y explique el comportamiento.\\
        \textbf{\color{red} RESPUESTA:}\\
        \item Esta estructura se deduce a partir de las reglas locales?.\\
        \textbf{\color{red} RESPUESTA:}\\
    \end{enumerate}
    
\end{enumerate}

\end{document}