\documentclass[12pt]{article}

\usepackage[spanish]{babel}
\usepackage[utf8]{inputenc}
\usepackage{graphicx}
\usepackage{geometry}
\usepackage{xcolor}
\usepackage{fancyhdr}
\usepackage{lastpage}
\usepackage{pdfpages}
\usepackage{listings}

\geometry{top=25mm,left=15mm,right=15mm,a4paper}

\pagestyle{fancy}
\fancyhf{}
\lhead{Lenguajes de Programación}
\cfoot{Página \thepage\ de \pageref{LastPage}}

\graphicspath{./}

\begin{document}
\includepdf{Portada.pdf}
{\color{red} \section*{Practica 2: Tema de Proyecto Final.}}

%Presentacion.pdf: Es el planteamiento e información para su proyecto
%final: debe contener: tema, objetivo, descripción del modelo y
%bibliografía

{\color{blue} \subsection*{Parte 3. Tema de proyecto final }}
\vspace{1em}

A lo largo del curso se han revisado varios sistemas modelados con Autómatas Celulares o Modelación Basada en Agentes.\\

Para realizar el proyecto final deberán escoger un tema de su interés sustentado en publicaciones académicas (libros o artículos) para
cumplir el siguiente objetivo:\\

\begin{enumerate}
    \item Replicar algunos de los resultados de algún artículo científico donde se use modelación basada en agentes.\\
    \item Extender el modelo a través de nuevas reglas, condiciones de entorno.\\
    \item Proponer un nuevo modelo.\\
\end{enumerate}

Para guiarse en el proceso se puede considerar:\\

\begin{enumerate}
    \item Realizar búsquedas en línea con el tema de su interés y modelos basados en agentes. Compartan las ligas de artículos en el Classroom para consulta de la clase.\\
    \item Estudiar y analizar el modelo. Estudiar las características del modelo, posibles estados de las celdas, reglas de evolución, inicialización del sistema, condiciones de frontera, etc.\\
    \item Observen en que lenguaje de programación esta implementado.\\
    \item Observen la inicialización del sistema, identifique los parámetros del sistema. ¿Qué pasa cuando se cambian los valores de los parametros?\\
    \item Observe el comportamiento global del sistema, ¿Tiene alguna propiedad emergente?.\\
    \item ¿Representa verdaderamente lo que pasa en realidad? el modelo explica el fenómeno o predice algun tipo de comportamiento. 
\end{enumerate}



{\color{blue} \subsubsection*{Ejercicios:}}



\end{document}